% easy/easy.tex
% SPDX-License-Identifier: CC-BY-SA-3.0

\QuickQuizChapter{chp:Ease of Use}{Ease of Use}
%
\Epigraph{Creating a perfect API is like committing the perfect crime.
	  There are at least fifty things that can go wrong, and if you are
	  a genius, you might be able to anticipate twenty-five of them.}
	 {\emph{With apologies to any Kathleen Turner fans who might
	  still be alive.}}

\section{What is Easy?}
\label{sec:easy:What is Easy?}

``Easy'' is a relative term.
For example, many people would consider a 15-hour airplane flight to be
a bit of an ordeal---unless they stopped to consider alternative modes
of transportation, especially swimming.
This means that creating an easy-to-use API requires that you know
quite a bit about your intended users.

The following question illustrates this point: ``Given a randomly chosen
person among everyone alive today, what one change would
improve his or her life?''

There is no single change that would be guaranteed to help everyone's life.
After all, there is an extremely wide range of people, with a correspondingly
wide range of needs, wants, desires, and aspirations.
A starving person might need food, but additional food might well hasten
the death of a morbidly obese person.
The high level of excitement so fervently desired by many young people
might well be fatal to someone recovering from a heart attack.
Information critical to the success of one person might contribute to
the failure of someone suffering from information overload.
In short, if you are working on a software project that is intended to
help someone you know nothing about, you should not be surprised when
that someone is less than impressed with your efforts.

If you really want to help a given group of people, there is simply no
substitute for working closely with them over an extended period of time.
Nevertheless, there are some simple things that you can do to increase
the odds of your users being happy with your software, and some of these
things are covered in the next section.

\section{Rusty Scale for API Design}
\label{sec:easy:Rusty Scale for API Design}

% Rusty is OK with this: July 19, 2006.

This section is adapted from portions of Rusty Russell's 2003 Ottawa Linux
Symposium keynote address~\cite[Slides 39-57]{RustyRussell2003OLSkeynote}.
Rusty's key point is that the goal should not be merely to make an API
easy to use, but rather to make the API hard to misuse.
To that end, Rusty proposed his ``Rusty Scale'' in decreasing order
of this important hard-to-misuse property.

The following list attempts to generalize the Rusty Scale beyond the
Linux kernel:

\begin{enumerate}
\item	It is impossible to get wrong.
	Although this is the standard to which all API designers should
	strive, only the mythical \co{dwim()}\footnote{
		The \co{dwim()} function is an acronym that expands to
		``do what I mean''.}
	command manages to come close.
\item	The compiler or linker won't let you get it wrong.
\item	The compiler or linker will warn you if you get it wrong.
\item	The simplest use is the correct one.
\item	The name tells you how to use it.
\item	Do it right or it will always break at runtime.
\item	Follow common convention and you will get it right.
	The \co{malloc()} library function is a good example.
	Although it is easy to get memory allocation wrong, a
	great many projects do manage to get it right, at least most
	of the time.
	Using \co{malloc()} in conjunction with
	Valgrind~\cite{ValgrindHomePage} moves \co{malloc()}
	almost up to the ``do it right or it will always break at runtime''
	point on the scale.
\item	Read the documentation and you will get it right.
\item	Read the implementation and you will get it right.
\item	Read the right mailing-list archive and you will get it right.
\item	Read the right mailing-list archive and you will get it wrong.
\item	Read the implementation and you will get it wrong.
	The original non-\co{CONFIG_PREEMPT} implementation of
	\co{rcu_read_lock()}~\cite{PaulEMcKenney2007PreemptibleRCU}
	is an infamous example of this point on the scale.
\item	Read the documentation and you will get it wrong.
	For example, the DEC Alpha \co{wmb} instruction's
	documentation~\cite{ALPHA2002} fooled a
	number of developers into thinking that that this instruction
	had much stronger memory-order semantics than it actually does.
	Later documentation clarified this
	point~\cite{Compaq01,WilliamPugh2000Gharachorloo},
	moving the \co{wmb} instruction up to the
	``read the documentation and you will get it right'' point on
	the scale.
\item	Follow common convention and you will get it wrong.
	The \co{printf()} statement is an example of this point on the
	scale because
	developers almost always fail to check \co{printf()}'s error return.
\item	Do it right and it will break at runtime.
\item	The name tells you how not to use it.
\item	The obvious use is wrong.
	The Linux kernel \co{smp_mb()} function is an example of
	this point on the scale.
	Many developers assume that this function has much
	stronger ordering semantics than it possesses.
	Chapter~\ref{chp:Advanced Synchronization: Memory Ordering} contains the
	information needed to avoid this mistake, as does the
	Linux-kernel source tree's \co{Documentation} directory.
\item	The compiler or linker will warn you if you get it right.
\item	The compiler or linker won't let you get it right.
\item	It is impossible to get right.
	The \co{gets()} function is a famous example of this point on
	the scale.
	In fact, \co{gets()} can perhaps best be described as
	an unconditional buffer-overflow security hole.
\end{enumerate}

\section{Shaving the Mandelbrot Set}
\label{sec:easy:Shaving the Mandelbrot Set}

The set of useful programs resembles the Mandelbrot set
(shown in Figure~\ref{fig:easy:Mandelbrot Set})
in that it does
not have a clear-cut smooth boundary---if it did, the halting problem
would be solvable.
But we need APIs that real people can use, not ones that require a
Ph.D. dissertation be completed for each and every potential use.
So, we ``shave the Mandelbrot set'',\footnote{
	Due to Josh Triplett.}
restricting the use of the
API to an easily described subset of the full set of potential uses.

\begin{figure}[htb]
\centering
\resizebox{3in}{!}{\includegraphics{easy/Mandel_zoom_00_mandelbrot_set}}
\caption{Mandelbrot Set (Courtesy of Wikipedia)}
\label{fig:easy:Mandelbrot Set}
\end{figure}

Such shaving may seem counterproductive.
After all, if an algorithm works, why shouldn't it be used?

To see why at least some shaving is absolutely necessary, consider
a locking design that avoids deadlock, but in perhaps the worst possible way.
This design uses a circular doubly linked list, which contains one
element for each thread in the system along with a header element.
When a new thread is spawned, the parent thread must insert a new
element into this list, which requires some sort of synchronization.

One way to protect the list is to use a global lock.
However, this might be a bottleneck if threads were being created and
deleted frequently.\footnote{
	Those of you with strong operating-system backgrounds, please
	suspend disbelief.
	If you are unable to suspend disbelief, send us a better example.}
Another approach would be to use a hash table and to lock the individual
hash buckets, but this can perform poorly when scanning the list in order.

A third approach is to lock the individual list elements, and to require
the locks for both the predecessor and successor to be held during the
insertion.
Since both locks must be acquired, we need to decide which order to
acquire them in.
Two conventional approaches would be to acquire the locks in address
order, or to acquire them in the order that they appear in the list,
so that the header is always acquired first when it is one of the two
elements being locked.
However, both of these methods require special checks and branches.

The to-be-shaven solution is to unconditionally acquire the locks in
list order.
But what about deadlock?

Deadlock cannot occur.

To see this, number the elements in the list starting with zero for the
header up to $N$ for the last element in the list (the one preceding the
header, given that the list is circular).
Similarly, number the threads from zero to $N-1$.
If each thread attempts to lock some consecutive pair of elements,
at least one of the threads is guaranteed to be able to acquire both
locks.

Why?

Because there are not enough threads to reach all the way around the list.
Suppose thread~0 acquires element~0's lock.
To be blocked, some other thread must have already acquired element~1's
lock, so let us assume that thread~1 has done so.
Similarly, for thread~1 to be blocked, some other thread must have acquired
element~2's lock, and so on, up through thread~$N-1$, who acquires
element~$N-1$'s lock.
For thread~$N-1$ to be blocked, some other thread must have acquired
element~$N$'s lock.
But there are no more threads, and so thread~$N-1$ cannot be blocked.
Therefore, deadlock cannot occur.

So why should we prohibit use of this delightful little algorithm?

The fact is that if you \emph{really} want to use it, we cannot stop you.
We \emph{can}, however, recommend against such code being included
in any project that we care about.

But, before you use this algorithm, please think through the following
Quick Quiz.

\QuickQuiz{}
	Can a similar algorithm be used when deleting elements?
\QuickQuizAnswer{
	Yes.
	However, since each thread must hold the locks of three
	consecutive elements to delete the middle one, if there
	are $N$ threads, there must be $2N+1$ elements (rather than
	just $N+1$) in order to avoid deadlock.
} \QuickQuizEnd

The fact is that this algorithm is extremely specialized (it only works
on certain sized lists), and also quite fragile.
Any bug that accidentally failed to add a node to the list could result
in deadlock.
In fact, simply adding the node a bit too late could result in deadlock.

In addition, the other algorithms described above are ``good and sufficient''.
For example, simply acquiring the locks in address order is fairly simple
and quick, while allowing the use of lists of any size.
Just be careful of the special cases presented by empty lists and lists
containing only one element!

\QuickQuiz{}
	Yetch!
	What ever possessed someone to come up with an algorithm
	that deserves to be shaved as much as this one does???
\QuickQuizAnswer{
	That would be Paul.

	He was considering the \emph{Dining Philosopher's Problem}, which
	involves a rather unsanitary spaghetti dinner attended by
	five philosophers.
	Given that there are five plates and but five forks on the table, and
	given that each philosopher requires two forks at a time to eat,
	one is supposed to come up with a fork-allocation algorithm that
	avoids deadlock.
	Paul's response was ``Sheesh!  Just get five more forks!''

	This in itself was OK, but Paul then applied this same solution to
	circular linked lists.

	This would not have been so bad either, but he had to go and tell
	someone about it!
} \QuickQuizEnd

In summary, we do not use algorithms simply because they happen to work.
We instead restrict ourselves to algorithms that are useful enough to
make it worthwhile learning about them.
The more difficult and complex the algorithm, the more generally useful
it must be in order for the pain of learning it and fixing its bugs to
be worthwhile.

\QuickQuiz{}
	Give an exception to this rule.
\QuickQuizAnswer{
	One exception would be a difficult and complex algorithm that
	was the only one known to work in a given situation.
	Another exception would be a difficult and complex algorithm
	that was nonetheless the simplest of the set known to work in
	a given situation.
	However, even in these cases, it may be very worthwhile to spend
	a little time trying to come up with a simpler algorithm!
	After all, if you managed to invent the first algorithm
	to do some task, it shouldn't be that hard to go on to
	invent a simpler one.
} \QuickQuizEnd

Exceptions aside, we must continue to shave the software ``Mandelbrot
set'' so that our programs remain maintainable, as shown in
Figure~\ref{fig:easy:Shaving the Mandelbrot Set}.

\begin{figure}[htb]
\centering
\resizebox{3in}{!}{\includegraphics{cartoons/r-2014-shaving-the-mandelbrot}}
\caption{Shaving the Mandelbrot Set}
\ContributedBy{Figure}{fig:easy:Shaving the Mandelbrot Set}{Melissa Broussard}
\end{figure}
