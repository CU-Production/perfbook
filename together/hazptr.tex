% together/hazptr.tex
% mainfile: ../perfbook.tex
% SPDX-License-Identifier: CC-BY-SA-3.0

\section{Hazard-Pointer Helpers}
\label{sec:together:Hazard-Pointer Helpers}
%
\epigraph{It's the little things that count, hundreds of them.}
	 {\emph{Cliff Shaw}}

This section looks at some issues that can arise when dealing with
hash tables.
Please note that these issues also apply to many other search structures.

\subsection{Scalable Reference Count}
\label{sec:together:Scalable Reference Count}

Suppose a reference count is becoming a performance or scalability
bottleneck.
What can you do?

One approach is to instead use hazard pointers.

There are some differences, perhaps most notably that with
hazard pointers it is extremely expensive to determine when
the corresponding reference count has reached zero.

One way to work around this problem is to split the load between
reference counters and hazard pointers.
Each data element has a reference counter that tracks the number
of other data elements referencing this element on the one hand,
and readers use hazard pointers on the other.

Making this arrangement work both efficiently and correctly can be
quite challenging, and so interested readers are invited to examine
the UnboundedQueue and ConcurrentHashMap data structures implemented in
Folly open-source library.\footnote{
	\url{https://github.com/facebook/folly}}

% @@@ Minimize memory footprint?
