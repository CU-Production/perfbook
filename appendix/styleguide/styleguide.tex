% appendix/styleguide/styleguide.tex
% SPDX-License-Identifier: CC-BY-SA-3.0

\chapter{Style Guide}
\label{chp:app:styleguide:Style Guide}

This appendix is a collection of style guides which is intended
as a reference to improve consistency in perfbook. It also contains
several suggestions and their experimental examples.

Section~\ref{sec:app:styleguide:Paul's Conventions} describes basic
punctuation and spelling rules.
Section~\ref{sec:app:styleguide:NIST Style Guide} explains rules
related to unit symbols.
Section~\ref{sec:app:styleguide:LaTeX Conventions} summarizes
\LaTeX-specific conventions.

\section{Paul's Conventions}
\label{sec:app:styleguide:Paul's Conventions}

Following is the list of Paul's conventions assembled from his
answers to Akira's questions regarding perfbook's punctuation policy.

\begin{itemize}
\item (On punctuations and quotations)
  Despite being American myself, for this sort of book, the UK approach
  is better because it removes ambiguities like the following:
  \begin{quote}
    Type ``\nbco{ls -a},'' look for the file ``\co{.},''
    and file a bug if you don't see it.
  \end{quote}

  The following is much more clear:
  \begin{quote}
    Type ``\nbco{ls -a}'', look for the file ``\co{.}'',
    and file a bug if you don't see it.
  \end{quote}
\item American English spelling: ``color'' rather than ``colour''.
\item Oxford comma: ``a, b, and~c'' rather than ``a, b and~c''.
  This is arbitrary.  Cases where the Oxford comma results in ambiguity
  should be reworded, for example, by introducing numbering:  ``a,
  b, and c and~d'' should be ``(1)~a, (2)~b, and (3)~c and~d''.
\item Italic for emphasis.  Use sparingly.
\item \verb|\co{}| for identifiers, \verb|\url{}| for URLs,
  \verb|\path{}| for filenames.
\item Dates should use an unambiguous format.  Never ``mm/dd/yy''
  or ``dd/mm/yy'', but rather ``July 26, 2016'' or ``26 July 2016''
  or ``26-Jul-2016'' or ``2016/07/26''.  I tend to use
  \path{yyyy.mm.ddA} for filenames, for example.
\item North American rules on periods and abbreviations.
  For example neither of the following can reasonably be interpreted
  as two sentences:
  \begin{itemize}
  \item Say hello, to Mr. Jones.
  \item If it looks like she sprained her ankle, call Dr. Smith and
    then tell her to keep the ankle iced and elevated.
  \end{itemize}

  An ambiguous example:
  \begin{quote}
    If I take the cow, the pig, the horse, etc. George will be upset.
  \end{quote}
  can be written with more words:
  \begin{quote}
    If I take the cow, the pig, the horse, or much of anything else,
    George will be upset.
  \end{quote}
  or:
  \begin{quote}
    If I take the cow, the pig, the horse, etc., George will be upset.
  \end{quote}
\item I don't like ampersand (``\&'') in headings, but will sometimes
  use it if doing so prevents a line break in that heading.
\item When mentioning words, I use quotations.  When introducing
  a new word, I use \verb|\emph{}|.
\end{itemize}

\section{NIST Style Guide}
\label{sec:app:styleguide:NIST Style Guide}

\subsection{Unit Symbol}
\label{sec:app:styleguide:Unit Symbol}

\subsubsection{SI Unit Symbol}
\label{sec:app:styleguide:SI Unit Symbol}

NIST style guide\footnote{
  \url{https://www.nist.gov/pml/nist-guide-si-chapter-7-rules-and-style-conventions-expressing-values-quantities}}
states the following rules (rephrased for perfbook).
% cite: https://www.nist.gov/pml/nist-guide-si-chapter-7-rules-and-style-conventions-expressing-values-quantities

\begin{itemize}
\item When SI unit symbols such as ``ns'', ``MHz'', and ``K'' (kelvin)
are used behind numerical values, narrow spaces should be placed between
the values and the symbols.

A narrow space can be coded in \LaTeX{} by the sequence of
\qco{\\,}.
For example,
\begin{quote}
  ``2.4\,GHz'', rather then ``2.4GHz''.
\end{quote}

\item Even when the value is used in adjectival sense, a narrow space
should be placed. For example,
\begin{quote}
  ``a~10\,ms interval'', rather than ``a~10\=/ms interval'' nor
  ``a~10ms interval''.
\end{quote}
\end{itemize}

The symbol of micro (\micro :$10^{-6}$) can be typeset easily by
the help of ``gensymb'' \LaTeX\ package.
A macro \qco{\\micro} can be used in both text and
math modes. To typeset the symbol of ``microsecond'', you can do
so by \qco{\\micro s}. For example,
\begin{quote}
  10\,\micro s
\end{quote}

Note that math mode \qco{\\mu} is italic by default and should not
be used as a prefix. An improper example:
\begin{quote}
  10\,$\mu $s (math mode \qco{\\mu})
\end{quote}

\subsubsection{Non-SI Unit Symbol}
\label{sec:app:styleguide:Non-SI Unit Symbol}

Although NIST style guide does not cover non-SI unit symbols
such as ``KB'', ``MB'', and ``GB'', the same rule should be followed.

Example:

\begin{quote}
  ``A~240\,GB hard drive'', rather than ``a~240\=/GB hard drive''
  nor ``a~240GB hard drive''.
\end{quote}

Strictly speaking, NIST guide requires us to use the binary prefixes
``Ki'', ``Mi'', or ``Gi'' to represent powers of~$2^{10}$.
However, we accept the JEDEC conventions to use ``K'', ``M'',
and ``G'' as binary prefixes in describing memory capacity.\footnote{
  \url{https://www.jedec.org/standards-documents/dictionary/terms/mega-m-prefix-units-semiconductor-storage-capacity}}

An acceptable example:
\begin{quote}
  ``8\,GB of main memory'', meaning ``8\,GiB of main memory''.
\end{quote}

Also, it is acceptable to use just ``K'', ``M'', or ``G'' as abbreviations
appended to a numerical value, e.g., ``4K~entries''. In such cases, no space
before an abbreviation is required. For example,

\begin{quote}
  ``8K entries'', rather than ``8\,K entries''.
\end{quote}

If you put a space in between, the symbol looks like a unit symbol and
is confusing.
Note that ``K'' and ``k'' represent $2^{10}$ and $10^3$, respectively.
``M'' can represent either $2^{20}$ or $10^6$, and ``G'' can represent
either $2^{30}$ or $10^9$. These ambiguities should not be confusing
in discussing approximate order.

\subsubsection{Degree Symbol}
\label{sec:app:styleguide:Degree Symbol}

The angular-degree symbol (\degree) does not require any space in front
of it. NIST style guide clearly states so.

The symbol of degree can also be typeset easily by the help of gensymb
package.
A macro \qco{\\degree} can be used in both text and math modes.

Example:

\begin{quote}
  $45\degree$, rather than $45\,\degree$.
\end{quote}

\subsection{NIST Guide Yet To Be Followed}
\label{sec:app:styleguide:NIST Guides Yet To Be Followed}

There are a few cases where NIST style guide is not followed.
Other English conventions are preferred in such cases.

\subsubsection{Digit Spacing}
\label{sec:app:styleguide:Digit Spacing}

Quote from NIST check list:\footnote{
  \#16 in \url{http://physics.nist.gov/cuu/Units/checklist.html}.
}

\begin{quote}
  The digits of numerical values having more than four digits on either
  side of the decimal marker are separated into groups of three using
  a thin, fixed space counting from both the left and right of the decimal
  marker. Commas are not used to separate digits into groups of three.
\end{quote}

\begin{quote}
\begin{tabular}{ll}
  NIST Example:& 15\,739.012\,53\,ms\\
  Our conventions:& 15,739.01253\,ms\\
\end{tabular}
\end{quote}

In \LaTeX\ coding, it is cumbersome to place thin spaces as are recommended
in NIST guide. The \verb|\num{}| command provided by the ``siunitx''
package would be of help for us to follow this rule.

\subsubsection{Percent Symbol}
\label{sec:app:styleguide:Percent Symbol}

NIST style guide treats the percent symbol (\%) as the same as SI unit
symbols.
In this textbook, no space is required in front of a percent symbol.

\begin{quote}
\begin{tabular}{ll}
  NIST guide:& 50\,\% possibility\\
  Our conventions:& 50\% possibility\\
\end{tabular}
\end{quote}

\subsubsection{Font Style}
\label{sec:app:styleguide:Font Style}

Quote from NIST check list:\footnote{
  \#6 in \url{https://physics.nist.gov/cuu/Units/checklist.html}
}

\begin{quote}
  Variables and quantity symbols are in italic type. Unit symbols
  are in roman type. Numbers should generally be written in roman
  type. These rules apply irrespective of the typeface used in
  the surrounding text.
\end{quote}

For example,
\begin{quote}
  {\textit e} (elementary charge)
\end{quote}

On the other hand, mathematical constants such as the base
of natural logarithms should be roman.\footnote{
  \url{https://physics.nist.gov/cuu/pdf/typefaces.pdf}
}
For example,

\begin{quote}
  $\mathrm{e}^x$
\end{quote}

In this textbook, this rule is not much considered as of this writing.
Most letters in math mode are italic regardless of what they
represent. Exceptions are uppercase Greek letters, which are upright
in math mode by default.\footnote{
  See \url{https://tex.stackexchange.com/questions/119248/}
  for the historical reason.}

\pagebreak % to prevent footnotes from spilling into next page
\section{\LaTeX\ Conventions}
\label{sec:app:styleguide:LaTeX Conventions}

Good looking \LaTeX\ documents require further considerations
on proper use of font styles, line break exceptions, etc.
This section summarizes guidelines specific to \LaTeX.

\subsection{Monospace Font}
\label{sec:app:styleguide:Monospace Font}

Monospace font (or typewriter font) is heavily used in this textbook.
First policy regarding monospace font in perfbook is to avoid
directly using \qco{\\texttt} or \qco{\\tt} macro.
It is highly recommended to use a macro or an environment
indicating the reason why you want the font.

This section explains the use cases of such macros and environments.

\subsubsection{Code Snippet}
\label{sec:app:styleguide:Code Snippet}

Although the ``verbatim'' environment is commonly used to include
listings, we use the ``verbbox'' environment provided by the
``verbatimbox'' package for most code snippets to enable the
centering layout.

% Another option would be the ``lstlisting'' environment provided
%  by the ``listings'' package. We are already using its ``lstinline''
%  command in the definition of \co{\\co\{\}} macro.

\begin{figure}[tbh]
{ \scriptsize
\verbfilebox[{\makebox[5ex][r]{\arabic{VerbboxLineNo}:\hspace{2ex}}}]
	{appendix/styleguide/samplecodesnippet.tex}
}
\centering
\theverbbox
\caption{\LaTeX\ Source of Sample Code Snippet}
\label{fig:app:styleguide:LaTeX Source of Sample Code Snippet}
\end{figure}

\begin{figure}[tbh]
{ \scriptsize
\begin{verbbox}
  1 /*
  2  * Sample Code Snippet
  3  */
  4  #include <stdio.h>
  5  int main(void)
  6  {
  7    printf("Hello world!\n");
  8    return 0;
  9  }
\end{verbbox}
}
\centering
\theverbbox
\caption{Sample Code Snippet}
\label{fig:app:styleguide:Sample Code Snippet}
\end{figure}


The \LaTeX\ source of a sample code snippet is shown in
Figure~\ref{fig:app:styleguide:LaTeX Source of Sample Code Snippet}
and is typeset as show in
Figure~\ref{fig:app:styleguide:Sample Code Snippet}.

Note that the verbbox environment is placed inside the figure environment.
This is to avoid a side effect of verbbox environment that interferes
with the ``afterheading\-/ness'' of a section's first sentence
when a verbbox is placed just below a heading.

Line numbers are manually placed for ease of referencing them within
\LaTeX\ sources.\footnote{
  As a matter of fact, the verbatimbox package has its own line
  number counter and the count can be displayed in the resulting
  listing. Leftmost line numbers in
  Figure~\ref{fig:app:styleguide:LaTeX Source of Sample Code Snippet}
  are output by the feature. See the source of this Section
  in \path{appendix/styleguide/styleguide.tex}
  if you are interested.}

The verbatim environment is used for listings with too many lines
to fit in a column. It is also used to avoid overwhelming
\LaTeX\ with a lot of floating objects.

\subsubsection{Identifier}
\label{sec:app:styleguide:Identifier}

We use ``\verb|\co{}|'' macro for inline identifiers.
(``co'' stands for ``code''.)

By putting them into \verb|\co{}|, underscore characters in
their names are free of escaping in \LaTeX\ source. It is convenient
to search them in source files. Also, \verb|\co{}|
macro has a capability to permit line breaks at particular
sequences of letters. Current definition permits a line break at
an underscore (\tco{_}), two consecutive underscores (\tco{__}),
a white space, or an operator \tco{->}.

\subsubsection{Identifier inside Table and Heading}
\label{sec:app:styleguide:Identifier inside Table and Heading}

Although \verb|\co{}| command is convenient for inlining within text,
it is fragile because of its capability of line break.
When it is used inside a ``tabular'' environment or its derivative
such as ``tabulary'', it confuses column width
estimation of those environments.
Furthermore, \verb|\co{}| can not be safely used in section headings nor
description headings.

As a workaround, we use ``\verb|\tco{}|'' command
inside tables and headings. It has no capability of line break
at particular sequences, but still frees us from escaping
underscores.

When used in text, \verb|\tco{}| permits line breaks at
white spaces.

\subsubsection{Other Use Cases of Monospace Font}
\label{sec:app:styleguide:Other Use Cases of Monospace Font}

For URLs, we use ``\verb|\url{}|'' command provided by the
``hyperref'' package. It will generate hyper references to the
URLs.

For path names, we use ``\verb|\path{}|'' command. It won't
generate hyper references.

Both \verb|\url{}| and \verb|\path{}| permit line breaks
at \qco{/}, \qco{-}, and \qco{.}.\footnote{
  Overfill can be a problem if the URL or the path name contains
  long runs of unbreakable characters.
}

For short monospace statements not to be line broken, we use
the ``\verb|\nbco{}|'' (non-breakable co) macro.

\subsubsection{Limitations}
\label{sec:app:styleguide:Limitations}

There are a few cases where macros introduced in this section
do not work as expected.
Table~\ref{tab:app:styleguide:Limitation of Monospace Macro}
lists such limitations.

\begin{table}[tbh]
\centering\footnotesize
\begin{tabular}{lll}
  Macro &  Need Escape & Should Avoid \\
  \hline
  \co{\\co}, \co{\\nbco} & \co{\\}, \%, \{, \} & \\
  \co{\\tco}  & \# & \%, \{, \}, \co{\\} \\
\end{tabular}
\caption{Limitation of Monospace Macro}
\label{tab:app:styleguide:Limitation of Monospace Macro}
\end{table}

While \verb|\co{}| requires some characters to be escaped,
it can contain any character.

On the other hand, \verb|\tco{}| can not handle
\qco{\%}, \qco{\{}, \qco{\}}, nor \qco{\\} properly.
If they are escaped by a~\qco{\\},
they appear in the end result with the escape character.
The \qco{\\verb} macro can be used in running text if you
need to use monospace font for a string which contains
many characters to escape.\footnote{
  \co{\\verb} is not almighty though. For example, you can't use it
  within a footnote nor table inside ``floatrow'' environment mentioned
  later. If you do so, you will see a fatal latex error.
  There are several workarounds of this problem, but as for perfbook,
  \co{\\co\{\}} should suffice.}

\subsection{Non Breakable Spaces}
\label{sec:app:styleguide:Non Breakable Spaces}

In \LaTeX\ conventions, proper use of non-breakable white spaces
is highly recommended. They can prevent widowing and orphaning
of single digit numbers or short variable names, which would
cause the text to be confusing at first glance.

The thin space mentioned earlier to be placed in front of a unit
symbol is non breakable.

Other cases to use a non-breakable space (``\verb|~|'' in \LaTeX\
source, often referred to as ``nbsp'')
are the following (inexhaustive).

\begin{itemize}
\item Reference to a Chapter or a Section:
  \begin{quote}
    Please refer to Section~\ref{sec:app:styleguide:NIST Style Guide}.
  \end{quote}
\item Calling out CPU number or Thread name:
  \begin{quote}
    After they load the pointer, CPUs~1 and~2 will see the stored
    value.
  \end{quote}
\item Short variable name:
  \begin{quote}
    The results will be stored in variables~\co{a} and~\co{b}.
  \end{quote}
\end{itemize}

\subsection{Hyphenation and Dashes}
\label{sec:app:styleguide:Hyphenation and Dashes}

\subsubsection{Hyphenation in Compound Word}
\label{sec:app:styleguide:Hyphenation in Compound Word}

In plain \LaTeX, compound words such as ``high-frequency''
can be hyphenated only at the hyphen. This sometimes results
in poor typesetting. For example:

\begin{center}\begin{minipage}{2.55in}\vspace{0.6\baselineskip}
  High-frequency radio wave, high-frequency radio wave,
  high-frequency radio wave, high-frequency radio wave,
  high-frequency radio wave, high-frequency radio wave.
\vspace{0.6\baselineskip}\end{minipage}\end{center}

By using a shortcut \qco{\\-/} provided by the
``extdash'' package, hyphenation in elements of compound
words is enabled in perfbook.\footnote{
  In exchange for enabling the shortcut, we can't use plain
  \LaTeX's shortcut \qco{\\-} to specify hyphenation points.
  Use \path{pfhyphex.tex} to add such exceptions.
}

Example with \qco{\\-/}:

\begin{center}\begin{minipage}{2.55in}\vspace{0.6\baselineskip}
  High\-/frequency radio wave, high\-/frequency radio wave,
  high\-/frequency radio wave, high\-/frequency radio wave,
  high\-/frequency radio wave, high\-/frequency radio wave.
\vspace{0.6\baselineskip}\end{minipage}\end{center}

\subsubsection{Non Breakable Hyphen}
\label{sec:app:styleguide:Non Breakable Hyphen}

We want hyphenated compound terms such as ``x\=/coordinate'',
``y\=/coordinate'', etc. not to be broken at the hyphen
following a single letter.

To make a hyphen unbreakable, we can use a short cut
\qco{\\=/} also provided by the ``extdash'' package.

Example without a shortcut:

\begin{center}\begin{minipage}{2.55in}\vspace{0.6\baselineskip}
x-, y-, and z-coordinates; x-, y-, and z-coordinates;
x-, y-, and z-coordinates; x-, y-, and z-coordinates;
x-, y-, and z-coordinates; x-, y-, and z-coordinates;
\vspace{0.6\baselineskip}\end{minipage}\end{center}

Example with \qco{\\-/}:

\begin{center}\begin{minipage}{2.55in}\vspace{0.6\baselineskip}
x-, y-, and z\-/coordinates; x-, y-, and z\-/coordinates;
x-, y-, and z\-/coordinates; x-, y-, and z\-/coordinates;
x-, y-, and z\-/coordinates; x-, y-, and z\-/coordinates;
\vspace{0.6\baselineskip}\end{minipage}\end{center}

Example with \qco{\\=/}:

\begin{center}\begin{minipage}{2.55in}\vspace{0.6\baselineskip}
x-, y-, and z\=/coordinates; x-, y-, and z\=/coordinates;
x-, y-, and z\=/coordinates; x-, y-, and z\=/coordinates;
x-, y-, and z\=/coordinates; x-, y-, and z\=/coordinates;
\vspace{0.6\baselineskip}\end{minipage}\end{center}

Note that \qco{\\=/} enables hyphenation in elements
of compound words as the same as \qco{\\-/} does.

\subsubsection{Em Dash}
\label{sec:app:styleguide:Em Dash}

Em dashes are used to indicate parenthetic expression. In perfbook,
em dashes are placed without spaces around it. In \LaTeX\ source,
an em dash is represented by \qco{---}.

Example (quote from Section~\ref{sec:app:whymb:Cache Structure}):
\begin{quote}
  This disparity in speed---more than two orders of magnitude---has
  resulted in the multi-megabyte caches found on modern CPUs.
\end{quote}

\subsubsection{En Dash}
\label{sec:app;styleguide:En Dash}

In \LaTeX\ convention, en~dashes (\==) are used for a range of (mostly)
numbers.
However, this is not followed in perfbook at all.
Because of the heavy use of dashes (\=/) for such cases
in plain-text communication, to make the \LaTeX\ sources compatible
with them, plain dashes are kept unmodified in the sources.

As a compromise, for those who are accustomed to en~dashes representing
ranges, there is a script to substitute en~dashes for plain dashes.

If you have the git repository of perfbook, by using a script
\path{utilities/dohyphen2endash.sh}, you can do the substitutions.
The script works only when you are in a clean git repository.
Otherwise it will just abort to prevent you from losing local
changes.

Example with a simple dash:

\begin{quote}
  Lines~4\=/12 in
  Figure~\ref{fig:app:styleguide:LaTeX Source of Sample Code Snippet}
  are the contents of the verbbox environment. The box is output
  by the \co{\\theverbbox} macro on line~16.
\end{quote}

Example with an en dash:

\begin{quote}
  Lines~4\==12 in
  Figure~\ref{fig:app:styleguide:LaTeX Source of Sample Code Snippet}
  are the contents of the verbbox environment. The box is output
  by the \co{\\theverbbox} macro on line~16.
\end{quote}

\subsubsection{Numerical Minus Sign}
\label{sec:app:styleguide:Numerical Minus Sign}

Numerical minus signs should be coded as math mode minus signs,
namely \qco{$-$}. For example,

\begin{quote}
  $-30$, rather than -30.
\end{quote}

\subsection{Improvement Candidates}
\label{sec:app:styleguide:Improvement Candidates}

There are a few areas yet to be attempted in perfbook
which would further improve its appearance.
This section lists up such candidates.

\subsubsection{Environment for Code Snippets}
\label{sec:app:styleguide:Environment for Code Snippets}

Strictly speaking, code snippets are \emph{not} figures.
They deserve their own floating environment.
The ``floatrow'' package would be of help.
Figure~\ref{fig:app:styleguide:Sample Code Snippet}
can be typeset as in
Listing~\ref{lst:app:styleguide:Sample Code Snippet}
using an experimental environment ``listing''.

\begin{listing}
{ \scriptsize
\verbfilebox[{\makebox[5ex][r]{\arabic{VerbboxLineNo}\hspace{2ex}}}]
	{appendix/styleguide/hello.c}
}
\centering
\theverbbox
\caption{Sample Code Snippet}
\label{lst:app:styleguide:Sample Code Snippet}
\end{listing}

\subsubsection{Position of Caption}
\label{sec:app:styleguide:Position of Caption}

In \LaTeX\ conventions, captions of tables should be placed
above them. The reason is the flow of your eye movement
when you look at them. Most tables have a row of heading at the
top. You naturally look at the top of a table at first. Captions at
the bottom of tables disturb this flow.
The same can be said of code snippets, which are read from
top to bottom.
The floatrow package mentioned above also has the capability
to adjust layout of caption.
For example,
Listing~\ref{lst:app:styleguide:Sample Code Snippet (Top)}
has the option \qco{cappostion=top} in its preamble.

\begin{listing}\RawFloats
{ \scriptsize
\verbfilebox[{\makebox[5ex][r]{\arabic{VerbboxLineNo}\hspace{2ex}}}]
	{appendix/styleguide/hello.c}
}
\begin{floatrow}[1]\thisfloatsetup{capposition=top}
  \floatbox{listing}[2.5in]{\caption{Sample Code Snippet (Top)}
    \label{lst:app:styleguide:Sample Code Snippet (Top)}}{
\theverbbox
}
\end{floatrow}
\end{listing}

For code snippets, the ``ruled'' style would look even better.
Listing~\ref{lst:app:styleguide:Sample Code Snippet (Ruled)}
is an example using the option \qco{style=ruled}.

\begin{listing}\RawFloats
{ \scriptsize
\verbfilebox[{\makebox[5ex][r]{\arabic{VerbboxLineNo}\hspace{2ex}}}]
	{appendix/styleguide/hello.c}
}
\begin{floatrow}[1]\thisfloatsetup{style=ruled}
  \floatbox{listing}[2.5in]{\caption{Sample Code Snippet (Ruled)}
    \label{lst:app:styleguide:Sample Code Snippet (Ruled)}}{
\theverbbox
}
\end{floatrow}
\end{listing}

Once the conversion of code sippets to a new environment has
completed, we would be able to choose one of the style options
as the default for the environment.

\subsubsection{Grouping Related Figures/Listings}
\label{sec:app:styleguide:Grouping Related Figures/Listings}

To prevent a pair of closely related figures or listings
from being placed in different pages, it is desirable to group
them into a single floating environment.
The floatrow package provides the features to do so.\footnote{
  Note that the ``subfigure'' package currently declared in the
  preamble is not used any more.
  The floatrow package is the choice to group floating environments
  these days.
  One problem of grouping figures might be the learning curve
  to do so.}

% TODO: Add example

\subsubsection{Ruled Line in Table}
\label{sec:app:styleguide:Ruled Line in Table}

They say that tables drawn by using ruled lines of plain \LaTeX\
look ugly.\footnote{
  \url{https://www.inf.ethz.ch/personal/markusp/teaching/guides/guide-tables.pdf}
}
Vertical lines should be avoided and horizontal lines should be
used sparingly, especially in tables of simple structure.

% TODO: Add example

\subsubsection{Miscellaneous Candidates}
\label{sec:app:styleguide:Miscellaneous Candidates}

Other improvement candidates are listed in the source of this
section as comments.

% Capitalize initialism:
%    Gnu Compiler Collection = GCC
%    gcc should be used as a command name in \co{gcc}
%    When mentioning GCC's C language, use `GNU C'
%
% Trademarks:
%    As the Legal page covers trademarks, there is no need to
%    use trademark symbol in the text. They seems to have been
%    imported from original publications.
%
% Power or POWER?
%    IBM's trademark page at https://www.ibm.com/legal/us/en/copytrade.shtml#section-P
%    lists the following.
%        PowerPC
%        Power Architecture
%        Power
%        POWER
%        POWER5
%        POWER6
%
%    not Power5, POWER 5, nor Power-5
%
% Ugly line break by \co{}
%                                 __
%        atomic_store()
%
%                           seqlock_
%        t
%
%   Is there any way to prevent these breaks?
%   Maybe we need an on-the-fly script to convert such \co{}s
%   to couples of \co{}s.
%   Example:
%     \co{__atomic_store()} -> \co{__}\co{atomic_store()}
%     \co{seqlock_t} ->        \co{seqlock_}\co{t}
