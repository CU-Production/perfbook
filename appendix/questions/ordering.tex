% appendix/questions/ordering.tex
% mainfile: ../../perfbook.tex
% SPDX-License-Identifier: CC-BY-SA-3.0

\section{How Much Ordering?}
\label{sec:app:questions:How Much Ordering?}

How much ordering is enough?

Perhaps you have carefully constructed a strongly ordered concurrent
system, only to find that it neither performs nor scales well.
Or perhaps you threw caution to the wind, only to find that your
brilliantly fast and scalable software is also unreliable.
Is there a happy medium with both robust reliability on the one
hand and powerful performance augmented by scintellating scalability on
the other?

The answer, as is so often the case, is ``it depends''.

One approach is to construct a strongly ordered system, then examine
its performance and scalablity.
If these suffice, the system is good and sufficient, and no more need
be done.
Otherwise, undertake careful analysis
(see \Cref{sec:debugging:Performance Estimation})
and attack the bottleneck located thereby.

This approach can work very well, especially in contrast to the
all-too-common approach of optimizing random components of the system
in the hope of achieving significant system-wide benefits.
However, starting with strong ordering can also be quite wasteful,
given that weakening ordering of the system's bottleneck can require
that large portions of the rest of the system be redesigned and
rewritten to accommodate the weakening.
Worse yet, eliminating one bottleneck often exposes another, which
in turn needs to be weakened and which in turn can result in wholesale
redesigns and rewrites of other parts of the system.
Perhaps even worse is the approach, also common, of starting with a
fast but unreliable system and then playing whack-a-mole with an endless
succession of concurrency bugs, though in the latter case,
Chapters~\ref{chp:Validation}
and~\ref{chp:Formal Verification}
are always there for you.

It would be better to have design-time tools to determine which portions
of the system can feature weak ordering, and at the same time, which
portions actually benefit from weak ordering.
These tasks are taken up by the following sections.

\subsection{Where is the Defining Data?}
\label{sec:app:questions:Where is the Defining Data?}

One way to do this is to keep firmly in mind that the region of
consistency engendered by strong ordering cannot extend out past the
boundaries of the system.\footnote{
	Which might well be a distributed system.}
Portions of the system whose role is to track the state of the outside
world can usually feature weak ordering, given that speed-of-light delays
will force the within-system state to lag behind the outside world.
There is often no point in incurring large overheads to force a consistent
view of data that is inherently out of date.
In these cases, the methods of \Cref{chp:Deferred Processing} can be
quite helpful, as can some of the data structures described in
\Cref{chp:Data Structures}.

Some might argue that useful computing deals only with the outside world,
and therefore that all computing can use weak ordering.
Such arguments are incorrect.
For example, the value of your bank account is defined within your bank's
computers, and people often prefer exact computations involving their
account balances, especially those cynical enough to suspect that any
approximating would be in the bank's favor.

In short, although data tracking external state can be an attractive
candidate for weakly ordered access, please think carefully about
exactly what is being tracked and what is doing the tracking.

\subsection{Consistent Data Used Consistently?}
\label{sec:app:questions:Consistent Data Used Consistently?}

Another hint that weakening is safe can appear in the guise of data
that is computed while holding a lock, but then used after the lock
is released.
The computed result clearly becomes at best an approximation as soon as
the lock is released, which suggests computing the result approximately
in the first place, possibly permitting use of weaker ordering.
To this end, \Cref{chp:Counting} covers numerous approximate methods
for counting.

Great care is required, however.
Is the use of data following lock release a hint that weak-ordering
optimizations might be helpful?
Or is instead a bug in which the lock was released too soon?

\subsection{Is the Problem Partitionable?}
\label{sec:app:questions:Is the Problem Partitionable?}

Suppose that the system holds the defining instance of the data,
or that using a computed value past lock release proved to be a bug.
What then?

One approach is to partition the system, as discussed in
\Cref{cha:Partitioning and Synchronization Design}.
Partititioning can provide excellent scalability and in its more
extreme form, per-CPU performance rivaling that of a sequential program,
as discussed in \Cref{chp:Data Ownership}.
Partial partitioning is often mediated by locking, which is the subject of
\Cref{chp:Locking}.

\subsection{None of the Above?}
\label{sec:app:questions:None of the Above?}

The previous sections described the easier ways to gain performance
and scalability, sometimes using weaker ordering and sometimes not.
But the plain fact is that multicore systems are under no compunction
to make life easy.
But perhaps the advanced topics covered in
Chapter~\ref{sec:advsync:Advanced Synchronization}
and~\ref{chp:Advanced Synchronization: Memory Ordering}
will prove helpful.

But please proceed with care, as it is all too easy to destabilize
your codebase optimizing non-bottlenecks.
Once again, \Cref{sec:debugging:Performance Estimation} can help.
It might also be worth your time to review other portions of this
book, as it contains much information on handling a number of tricky
situations.
