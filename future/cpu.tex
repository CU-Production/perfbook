% future/cpu.tex
% SPDX-License-Identifier: CC-BY-SA-3.0

\section{The Future of CPU Technology Ain't What it Used to Be}
\label{sec:future:The Future of CPU Technology Ain't What it Used to Be}

Years past always seem so simple and innocent when viewed through the
lens of many years of experience.
And the early 2000s were for the most part innocent of the impending
failure of Moore's Law to continue delivering the then-traditional
increases in CPU clock frequency.
Oh, there were the occasional warnings about the limits of technology,
but such warnings had been sounded for decades.
With that in mind, consider the following scenarios:

\begin{figure}[tb]
\centering
\resizebox{3in}{!}{\includegraphics{cartoons/r-2014-CPU-future-uniprocessor-uber-alles}}
\caption{Uniprocessor \"Uber Alles}
\ContributedBy{Figure}{fig:future:Uniprocessor Uber Alles}{Melissa Broussard}
\end{figure}

\begin{figure}[tb]
\centering
\resizebox{2.6in}{!}{\includegraphics{cartoons/r-2014-CPU-Future-Multithreaded-Mania}}
\caption{Multithreaded Mania}
\ContributedBy{Figure}{fig:future:Multithreaded Mania}{Melissa Broussard}
\end{figure}

\begin{figure}[tb]
\centering
\resizebox{2.5in}{!}{\includegraphics{cartoons/r-2014-CPU-Future-More-of-the-Same}}
\caption{More of the Same}
\ContributedBy{Figure}{fig:future:More of the Same}{Melissa Broussard}
\end{figure}

\begin{figure}[tb]
\centering
\resizebox{3in}{!}{\includegraphics{cartoons/r-2014-CPU-Future-Crash-dummies}}
\caption{Crash Dummies Slamming into the Memory Wall}
\ContributedBy{Figure}{fig:future:Crash Dummies Slamming into the Memory Wall}{Melissa Broussard}
\end{figure}

\begin{enumerate}
\item	Uniprocessor \"Uber Alles
	(Figure~\ref{fig:future:Uniprocessor Uber Alles}),
\item	Multithreaded Mania
	(Figure~\ref{fig:future:Multithreaded Mania}),
\item	More of the Same
	(Figure~\ref{fig:future:More of the Same}), and
\item	Crash Dummies Slamming into the Memory Wall
	(Figure~\ref{fig:future:Crash Dummies Slamming into the Memory Wall}).
\end{enumerate}

Each of these scenarios are covered in the following sections.

\subsection{Uniprocessor \"Uber Alles}
\label{sec:future:Uniprocessor Uber Alles}

As was said in 2004~\cite{PaulEdwardMcKenneyPhD}:

\begin{quote}
	In this scenario, the combination of Moore's-Law increases in CPU
	clock rate and continued progress in horizontally scaled computing
	render SMP systems irrelevant.
	This scenario is therefore dubbed ``Uniprocessor \"Uber
	Alles'', literally, uniprocessors above all else.

	These uniprocessor systems would be subject only to instruction
	overhead, since memory barriers, cache thrashing, and contention
	do not affect single-CPU systems.
	In this scenario, RCU is useful only for niche applications, such
	as interacting with NMIs.
	It is not clear that an operating system lacking RCU would see
	the need to adopt it, although operating
	systems that already implement RCU might continue to do so.

	However, recent progress with multithreaded CPUs seems to indicate
	that this scenario is quite unlikely.
\end{quote}

Unlikely indeed!
But the larger software community was reluctant to accept the fact that
they would need to embrace parallelism, and so it was some time before
this community concluded that the ``free lunch'' of Moore's-Law-induced
CPU core-clock frequency increases was well and truly finished.
Never forget: belief is an emotion, not necessarily the result of a
rational technical thought process!

\subsection{Multithreaded Mania}
\label{sec:future:Multithreaded Mania}

Also from 2004~\cite{PaulEdwardMcKenneyPhD}:

\begin{quote}
	A less-extreme variant of Uniprocessor \"Uber Alles features
	uniprocessors with hardware multithreading, and in fact
	multithreaded CPUs are now standard for many desktop and laptop
	computer systems.  The most aggressively multithreaded CPUs share
	all levels of cache hierarchy, thereby eliminating CPU-to-CPU
	memory latency, in turn greatly reducing the performance
	penalty for traditional synchronization mechanisms.  However,
	a multithreaded CPU would still incur overhead due to contention
	and to pipeline stalls caused by memory barriers.  Furthermore,
	because all hardware threads share all levels of cache, the
	cache available to a given hardware thread is a fraction of
	what it would be on an equivalent single-threaded CPU, which can
	degrade performance for applications with large cache footprints.
	There is also some possibility that the restricted amount of cache
	available will cause RCU-based algorithms to incur performance
	penalties due to their grace-period-induced additional memory
	consumption.  Investigating this possibility is future work.

	However, in order to avoid such performance degradation, a number
	of multithreaded CPUs and multi-CPU chips partition at least
	some of the levels of cache on a per-hardware-thread basis.
	This increases the amount of cache available to each hardware
	thread, but re-introduces memory latency for cachelines that
	are passed from one hardware thread to another.
\end{quote}

And we all know how this story has played out, with multiple multi-threaded
cores on a single die plugged into a single socket.
The question then becomes whether or not future shared-memory systems will
always fit into a single socket.

\subsection{More of the Same}
\label{sec:meas:More of the Same}

Again from 2004~\cite{PaulEdwardMcKenneyPhD}:

\begin{quote}
	The More-of-the-Same scenario assumes that the memory-latency
	ratios will remain roughly where they are today.

	This scenario actually represents a change, since to have more
	of the same, interconnect performance must begin keeping up
	with the Moore's-Law increases in core CPU performance.  In this
	scenario, overhead due to pipeline stalls, memory latency, and
	contention remains significant, and RCU retains the high level
	of applicability that it enjoys today.
\end{quote}

And the change has been the ever-increasing levels of integration
that Moore's Law is still providing.
But longer term, which will it be?
More CPUs per die?
Or more I/O, cache, and memory?

Servers seem to be choosing the former, while embedded systems on a chip
(SoCs) continue choosing the latter.

\subsection{Crash Dummies Slamming into the Memory Wall}
\label{sec:future:Crash Dummies Slamming into the Memory Wall}

\begin{figure}[tbp]
\centering
\epsfxsize=3in
\epsfbox{future/latencytrend}
% from Ph.D. thesis: related/latencytrend.eps
\caption{Instructions per Local Memory Reference for Sequent Computers}
\label{fig:future:Instructions per Local Memory Reference for Sequent Computers}
\end{figure}

\begin{figure}[htbp]
\centering
\epsfxsize=3in
\epsfbox{future/be-lb-n4-rf-all}
% from Ph.D. thesis: an/plots/be-lb-n4-rf-all.eps
\caption{Breakevens vs. $r$, $\lambda$ Large, Four CPUs}
\label{fig:future:Breakevens vs. r, lambda Large, Four CPUs}
\end{figure}

\begin{figure}[htbp]
\centering
\epsfxsize=3in
\epsfbox{future/be-lw-n4-rf-all}
% from Ph.D. thesis: an/plots/be-lw-n4-rf-all.eps
\caption{Breakevens vs. $r$, $\lambda$ Small, Four CPUs}
\label{fig:future:Breakevens vs. r, Worst-Case lambda, Four CPUs}
\end{figure}

And one more quote from 2004~\cite{PaulEdwardMcKenneyPhD}:

\begin{quote}
	If the memory-latency trends shown in
	Figure~\ref{fig:future:Instructions per Local Memory Reference for Sequent Computers}
	continue, then memory latency will continue to grow relative
	to instruction-execution overhead.
	Systems such as Linux that have significant use of RCU will find
	additional use of RCU to be profitable, as shown in
	Figure~\ref{fig:future:Breakevens vs. r, lambda Large, Four CPUs}
	As can be seen in this figure, if RCU is heavily used, increasing
	memory-latency ratios give RCU an increasing advantage over other
	synchronization mechanisms.
	In contrast, systems with minor
	use of RCU will require increasingly high degrees of read intensity
	for use of RCU to pay off, as shown in
	Figure~\ref{fig:future:Breakevens vs. r, Worst-Case lambda, Four CPUs}.
	As can be seen in this figure, if RCU is lightly used,
	increasing memory-latency ratios
	put RCU at an increasing disadvantage compared to other synchronization
	mechanisms.
	Since Linux has been observed with over 1,600 callbacks per grace
	period under heavy load~\cite{Sarma04c},
	it seems safe to say that Linux falls into the former category.
\end{quote}

On the one hand, this passage failed to anticipate the cache-warmth
issues that RCU can suffer from in workloads with significant update
intensity, in part because it seemed unlikely that RCU would really
be used for such workloads.
In the event, the \co{SLAB_TYPESAFE_BY_RCU} has been pressed into 
service in a number of instances where these cache-warmth issues would
otherwise be problematic, as has sequence locking.
On the other hand, this passage also failed to anticipate that
RCU would be used to reduce scheduling latency or for security.

In short, beware of prognostications, including those in the remainder
of this chapter.
