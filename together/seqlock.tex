% together/seqlock.tex
% mainfile: ../perfbook.tex
% SPDX-License-Identifier: CC-BY-SA-3.0

\section{Sequence-Locking Specials}
\label{sec:together:Sequence-Locking Specials}
%
\epigraph{The girl who can't dance says the band can't play.}
	 {\emph{Yiddish proverb}}

This section looks at some special uses of sequence counters.

\subsection{Correlated Data Elements}
\label{sec:together:Correlated Data Elements}

Suppose that have a hash table where we need correlated views of two or
more of the elements.
These elements are updated together, and we do not want to see an old
version of the first element along with new versions of the other
elements.
For example, Schr\"odinger decided to add his extended family to his
in-memory database along with all his animals.
Although Schr\"odinger understands that marriages and divorces do not
happen instantaneously, he is also a traditionalist.
As such, he absolutely does not want his database ever to show that the
bride is now married, but the groom is not, and vice versa.
Plus, if you think Schr\"odinger is a traditionalist, you just
try conversing with some of his family members!
In other words, Schr\"odinger wants to be able to carry out a
wedlock-consistent traversal of his database.

One approach is to use sequence locks
(see \cref{sec:defer:Sequence Locks}),
so that wedlock-related updates are carried out under the
protection of \co{write_seqlock()}, while reads requiring
wedlock consistency are carried out within
a \co{read_seqbegin()} / \co{read_seqretry()} loop.
Note that sequence locks are not a replacement for RCU protection:
Sequence locks protect against concurrent modifications, but RCU
is still needed to protect against concurrent deletions.

This approach works quite well when the number of correlated elements is
small, the time to read these elements is short, and the update rate is
low.
Otherwise, updates might happen so quickly that readers might never complete.
Although Schr\"odinger does not expect that even his least-sane relatives
will marry and divorce quickly enough for this to be a problem,
he does realize that this problem could well arise in other situations.
One way to avoid this reader-starvation problem is to have the readers
use the update-side primitives if there have been too many retries,
but this can degrade both performance and scalability.
Another way to avoid starvation is to have multiple sequence locks,
in Schr\"odinger's case, perhaps one per species.

In addition, if the update-side primitives are used too frequently,
poor performance and scalability will result due to lock contention.
One way to avoid this is to maintain a per-element sequence lock,
and to hold both spouses' locks when updating their marital status.
Readers can do their retry looping on either of the spouses' locks
to gain a stable view of any change in marital status involving both
members of the pair.
This avoids contention due to high marriage and divorce rates, but
complicates gaining a stable view of all marital statuses during a
single scan of the database.

If the element groupings are well-defined and persistent, which marital
status is hoped to be,
then one approach is to add pointers to the data elements to link
together the members of a given group.
Readers can then traverse these pointers to access all the data elements
in the same group as the first one located.

Other approaches using version numbering are left as exercises for the
interested reader.

Note that it is likely that similar schemes also work with hazard
pointers.
