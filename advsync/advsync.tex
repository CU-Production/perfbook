% advsync/advsync.tex

\QuickQuizChapter{sec:advsync:Advanced Synchronization}{Advanced Synchronization}
%
\Epigraph{If a little knowledge is a dangerous thing, just imagine all
	  the havoc you could wreak with a lot of knowledge!}{\emph{Unknown}}

This chapter discusses a number of ways of using weaker, and hopefully
lower-cost, synchronization primitives.
This weakening can be quite helpful, in fact, some have argued
that weakness is a virtue~\cite{JadeAlglave2013-WeaknessIsVirtue}.
Nevertheless, in parallel programming, as in many other aspects
of life, weakness is not a panacea.
For example, as noted at the end of
Chapter~\ref{chp:Counting},
you should thoroughly apply partitioning, batching, and
well-tested packaged weak APIs (see Chapter~\ref{chp:Data Ownership}
and~\ref{chp:Deferred Processing})
before even thinking about unstructured weakening.

But after doing all that, you still might find yourself needing the
advanced techniques described in this chapter.
To that end,
Section~\ref{sec:advsync:Avoiding Locks}
summarizes techniques used thus far for avoiding locks and
Section~\ref{sec:advsync:Non-Blocking Synchronization}
gives a brief overview of non-blocking synchronization.

\section{Avoiding Locks}
\label{sec:advsync:Avoiding Locks}

Although locking is the workhorse of parallelism in production, in
many situations performance, scalability, and real-time response can
all be greatly improved through use of lockless techniques.
A particularly impressive example of such a lockless technique are
the statistical counters described in
Section~\ref{sec:count:Statistical Counters},
which avoids not only locks, but also atomic operations, memory barriers,
and even cache misses for counter increments.
Other examples we have covered include:

\begin{enumerate}
\item	The fastpaths through a number of other counting algorithms
	in Chapter~\ref{chp:Counting}.
\item	The fastpath through resource allocator caches in
	Section~\ref{sec:SMPdesign:Resource Allocator Caches}.
\item	The maze solver in Section~\ref{sec:SMPdesign:Beyond Partitioning}.
\item	The data-ownership techniques described in
	Chapter~\ref{chp:Data Ownership}.
\item	The reference-counting and RCU techinques described in
	Chapter~\ref{chp:Deferred Processing}.
\item	The lookup code paths described in Chapter~\ref{chp:Data Structures}.
\item	Many of the techniques described in
	Chapter~\ref{chp:Putting It All Together}.
\end{enumerate}

In short, lockless techniques are quite useful and are heavily used.

However, it is best if lockless techniques are hidden behind a
well-defined API, such as the \co{inc_count()}, \co{memblock_alloc()},
\co{rcu_read_lock()}, and so on.
The reason for this is that undisciplined use of lockless techniques
is a good way to create difficult bugs.
If you don't believe that avoiding such bugs is easier than finding
and fixing them, please re-read
Chapters~\ref{chp:Validation} and~\ref{chp:Formal Verification}.

A key component of many lockless techniques is the memory barrier,
which is described in the following section.

\section{Non-Blocking Synchronization}
\label{sec:advsync:Non-Blocking Synchronization}

The term \emph{non-blocking synchronization} (NBS) describes six classes of
linearizable algorithms with differing \emph{forward-progress guarantees}.
These forward-progress guarantees are orthogonal to those that form the
basis of real-time programming:

\begin{enumerate}
\item	Real-time forward-progress guarantees usually have some
	definite time associated with them, for example,
	``scheduling latency must be less than 100 microseconds.''
	In contrast, the most popular forms of NBS only guarantees
	that progress will be made in finite time, with no definite
	bound.
\item	Real-time forward-progress guarantees are sometimes
	probabilistic, as in the soft-real-time guarantee that
	``at least 99.9\,\% of the time, scheduling latency must
	be less than 100 microseconds.''
	In contrast, NBS's forward-progress
	guarantees have traditionally been unconditional.
\item	Real-time forward-progress guarantees are often conditioned on
	environmental constraints, for example, only being honored
	for the highest-priority tasks, when each CPU spends at least
	a certain fraction of its time idle, or when I/O rates are
	below some specified maximum.
	In contrast, NBS's forward-progress
	guarantees are usually unconditional.\footnote{
		As we will see below, some recent NBS work relaxes
		this guarantee.}
\item	Real-time forward-progress guarantees usually apply only
	in the absence of software bugs.
	In contrast, most NBS guarantees apply even in the face of
	fail-stop bugs.\footnote{
		Again, some recent NBS work relaxes this guarantee.}
\item	NBS forward-progress guarantee classes imply linearizability.
	In contrast, real-time forward progress guarantees are often
	independent of ordering constraints such as linearizability.
\end{enumerate}

Despite these differences, a number of NBS algorithms are extremely
useful in real-time programs.

There are currently seven levels in the NBS
hierarchy~\cite{DanAlitarh2013PracticalProgress}, which are roughly
as follows:

\begin{enumerate}
\item	\emph{Bounded wait-free synchronization}: Every thread
	will make progress within
	a specific finite period of time~\cite{Herlihy91}.
	(Note that this level is 
	widely considered to be unachievable, which might be why
	Alitarh et al.~\cite{DanAlitarh2013PracticalProgress}
	omitted it.)
\item	\emph{Wait-free synchronization}: Every thread will make progress
	in finite time~\cite{Herlihy93}.
\item	\emph{Lock-free synchronization}: At least one thread will
	make progress in finite time~\cite{Herlihy93}.
\item	\emph{Obstruction-free synchronization}: Every thread will
	make progress in finite time in the absence of
	contention~\cite{HerlihyLM03}.
\item	\emph{Clash-free synchronization}: At least one thread will
	make progress in finite time in the absence of
	contention~\cite{DanAlitarh2013PracticalProgress}.
\item	\emph{Starvation-free synchronization}: Every thread will
	make progress in finite time in the absence of
	failures~\cite{DanAlitarh2013PracticalProgress}.
\item	\emph{Deadlock-free synchronization}: At least one thread will
	make progress in finite time in the absence of
	failures~\cite{DanAlitarh2013PracticalProgress}.
\end{enumerate}

NBS classes~1, 2 and~3 were first formulated in the early 1990s,
class~4 was first formulated in the early 2000s,
and class~5 was first formulated in 2013.
The final two classes have seen informal use for a great many decades,
but were reformulated in 2013.

In theory, any parallel algorithm can be cast into wait-free form,
but there are a relatively small subset of NBS algorithms that are
in common use.
A few of these are listed in the following section.

\subsection{Simple NBS}
\label{sec:advsync:Simple NBS}

Perhaps the simplest NBS algorithm is atomic update of an integer
counter using fetch-and-add (\co{atomic_add_return()}) primitives.

Another simple NBS algorithm implements a set of integers in an array.
Here the array index indicates a value that might be a member of the set
and the array element indicates whether or not that value actually is
a set member.
The linearizability criterion for NBS algorithms requires that reads from
and updates to the array either use atomic instructions or be accompanied
by memory barriers, but in the not-uncommon case where linearizability
is not important, simple volatile loads and stores suffice, for example,
using \co{ACCESS_ONCE()}.

An NBS set may also be implemented using a bitmap, where each value that
might be a member of the set corresponds to one bit.
Reads and updates must normally be carried out via atomic bit-manipulation
instructions, although compare-and-swap (\co{cmpxchg()} or CAS)
instructions can also be used.

The statistical counters algorithm discussed in
Section~\ref{sec:count:Statistical Counters}
can be considered wait-free, but only but using a cute definitional trick
in which the sum is considered approximate rather than exact.\footnote{
	Citation needed.
	I hear of this trick verbally from Mark Moir.}
Given sufficiently wide error bounds that are a function of the length
of time that the \co{read_count()} function takes to sum the counters,
it is not possible to prove that any non-linearizable behavior occurred.
This definitely (if a bit artificially) classifies the statistical-counters
algorithm as wait-free.
This algorithm is probably the most heavily used NBS algorithm in
the Linux kernel.

\begin{listing}[tbp]
{ \scriptsize
\begin{verbbox}
 1 static inline bool
 2 ___cds_wfcq_append(struct cds_wfcq_head *head,
 3                    struct cds_wfcq_tail *tail,
 4                    struct cds_wfcq_node *new_head,
 5                    struct cds_wfcq_node *new_tail)
 6 {
 7   struct cds_wfcq_node *old_tail;
 8 
 9   old_tail = uatomic_xchg(&tail->p, new_tail);
10   CMM_STORE_SHARED(old_tail->next, new_head);
11   return old_tail != &head->node;
12 }
13 
14 static inline bool
15 _cds_wfcq_enqueue(struct cds_wfcq_head *head,
16                   struct cds_wfcq_tail *tail,
17                   struct cds_wfcq_node *new_tail)
18 {
19   return ___cds_wfcq_append(head, tail,
20                             new_tail, new_tail);
21 }
\end{verbbox}
}
\centering
\theverbbox
\caption{NBS Enqueue Algorithm}
\label{lst:count:NBS Enqueue Algorithm}
\end{listing}

Another common NBS algorithm is the atomic queue where elements are
enqueued using an atomic exchange instruction~\cite{MagedMichael1993JPDC},
followed by a store into the \co{->next} pointer of the new element's
predecessor, as shown in
Listing~\ref{lst:count:NBS Enqueue Algorithm},
which shows the userspace-RCU library
implementation~\cite{MathieuDesnoyers2009URCU}.
Line~9 updates the tail pointer to reference the new element while
returning a reference to its predecessor, which is stored in
local variable \co{old_tail}.
Line~10 then updates the predecessor's \co{->next} pointer to
reference the newly added element, and finally line~11
returns an indication as to whether or not the queue was initially
empty.

Although mutual exclusion is required to dequeue a single element
(so that dequeue is blocking), it is possible to carry out a non-blocking
removal of the entire contents of the queue.
What is not possible is to dequeue any given element in a non-blocking
manner: The enqueuer might have failed between lines~9 and~10 of the
listing, so that the element in question is only partially enqueued.
This results in a half-NBS algorithm where enqueues are NBS but
dequeues are blocking.
This algorithm is nevertheless used in practice, in part because
most production software is not required to tolerate arbitrary fail-stop
errors.

\subsection{NBS Discussion}
\label{sec:advsync:NBS Discussion}

It is possible to create fully non-blocking queues~\cite{MichaelScott96},
however, such queues are much more complex than the half-NBS algorithm
outlined above.
The lesson here is to carefully consider what your requirements really are.
Relaxing irrelevant requirements can often result in great
improvements in both simplicity and performance.

Recent research points to another important way to relax requirements.
It turns out that systems providing fair scheduling can enjoy most
of the benefits of wait-free synchronization even when running
algorithms that provide only non-blocking
synchronization, both in theory~\cite{DanAlitarh2013PracticalProgress}
and in practice~\cite{SamyAlBahra2013NBS}.
Because a great many schedulers used in production do in fact
provide fairness,
the more-complex algorithms providing wait-free synchronization
usually provide no practical advantages over their simpler
and often faster non-blocking-synchronization counterparts.

Interestingly enough, fair scheduling is but one beneficial
constraint that is often respected in practice.
Other sets of constraints can permit blocking algorithms to
achieve deterministic real-time response.
For example, given fair locks that are granted to requesters in FIFO order at
a given priority level,
a method of avoiding priority inversion (such as priority
inheritance~\cite{Takada:1995:RSN:527074.828566,Cai-DongWang1996PrioInherLock}
or priority ceiling), a bounded number of threads,
bounded critical sections,
bounded load,
and avoidance of fail-stop bugs,
lock-based applications can provide deterministic
response times~\cite{BjoernBrandenburgPhD}.
% @@@ Check dissertation details once internet is available.
This approach of course blurs the distinction between blocking and wait-free
synchronization, which is all to the good.
Hopefully theoeretical frameworks continue to grow, further increasing
their ability to
describe how software is actually constructed in practice.
