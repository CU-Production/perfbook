% appendix/questions/buggy.tex
% mainfile: ../../perfbook.tex
% SPDX-License-Identifier: CC-BY-SA-3.0

\section{Why Is Software Buggy?}
\label{sec:app:questions:Why Is Software Buggy?}

The short answer is ``because it was written by humans, and to err
is human''.
This does not necessarily mean that automated code generation is
the answer, because the program that does the code generation will
have been written by humans.
In addition, one of the biggest problems in producing software is
working out what that software is supposed to do, and this task
has thus far proven rather resistant to automation.

Nevertheless, automation is an important part of the process of reducing
the number of bugs in software.
For but one example, despite their many flaws, it is almost always better
to use a compiler than to write in assembly language.

Furthermore, careful validation can be very helpful in finding bugs,
as discussed in
\crefrange{chp:Validation}{chp:Formal Verification}.
