% cpu/overheads.tex
% SPDX-License-Identifier: CC-BY-SA-3.0

\section{Overheads}
\label{sec:cpu:Overheads}
%
\epigraph{Don't design bridges in ignorance of materials, and don't design
	  low-level software in ignorance of the underlying hardware.}
	 {\emph{Unknown}}

This section presents actual overheads of the obstacles to performance
listed out in the previous section.
However, it is first necessary to get a rough view of hardware system
architecture, which is the subject of the next section.

\subsection{Hardware System Architecture}
\label{sec:cpu:Hardware System Architecture}

\begin{figure}[tb]
\centering
\resizebox{3in}{!}{\includegraphics{cpu/SystemArch}}
\caption{System Hardware Architecture}
\label{fig:cpu:System Hardware Architecture}
\end{figure}

Figure~\ref{fig:cpu:System Hardware Architecture}
shows a rough schematic of an eight-core computer system.
Each die has a pair of CPU cores, each with its cache, as well as an
interconnect allowing the pair of CPUs to communicate with each other.
The system interconnect in the middle of the diagram allows the
four dies to communicate, and also connects them to main memory.

Data moves through this system in units of ``cache lines'', which
are power-of-two fixed-size aligned blocks of memory, usually ranging
from 32 to 256 bytes in size.
When a CPU loads a variable from memory to one of its registers, it must
first load the cacheline containing that variable into its cache.
Similarly, when a CPU stores a value from one of its registers into
memory, it must also load the cacheline containing that variable into
its cache, but must also ensure that no other CPU has a copy of that
cacheline.

For example, if CPU~0 were to perform a compare-and-swap (CAS) operation on a
variable whose cacheline resided in CPU~7's cache, the following
over-simplified sequence of events might ensue:

\begin{enumerate}
\item	CPU~0 checks its local cache, and does not find the cacheline.
\item	The request is forwarded to CPU~0's and 1's interconnect,
	which checks CPU~1's local cache, and does not find the cacheline.
\item	The request is forwarded to the system interconnect, which
	checks with the other three dies, learning that the cacheline
	is held by the die containing CPU~6 and 7.
\item	The request is forwarded to CPU~6's and 7's interconnect, which
	checks both CPUs' caches, finding the value in CPU~7's cache.
\item	CPU~7 forwards the cacheline to its interconnect, and also
	flushes the cacheline from its cache.
\item	CPU~6's and 7's interconnect forwards the cacheline to the
	system interconnect.
\item	The system interconnect forwards the cacheline to CPU~0's and 1's
	interconnect.
\item	CPU~0's and 1's interconnect forwards the cacheline to CPU~0's
	cache.
\item	CPU~0 can now perform the CAS operation on the value in its cache.
\end{enumerate}

\QuickQuiz{}
	This is a \emph{simplified} sequence of events?
	How could it \emph{possibly} be any more complex?
\QuickQuizAnswer{
	This sequence ignored a number of possible complications,
	including:

	\begin{enumerate}
	\item	Other CPUs might be concurrently attempting to perform
		CAS operations involving this same cacheline.
	\item	The cacheline might have been replicated read-only in
		several CPUs' caches, in which case, it would need to
		be flushed from their caches.
	\item	CPU~7 might have been operating on the cache line when
		the request for it arrived, in which case CPU~7 might
		need to hold off the request until its own operation
		completed.
	\item	CPU~7 might have ejected the cacheline from its cache
		(for example, in order to make room for other data),
		so that by the time that the request arrived, the
		cacheline was on its way to memory.
	\item	A correctable error might have occurred in the cacheline,
		which would then need to be corrected at some point before
		the data was used.
	\end{enumerate}

	Production-quality cache-coherence mechanisms are extremely
	complicated due to these sorts of
	considerations~\cite{Hennessy95a,DavidECuller1999,MiloMKMartin2012scale,DanielJSorin2011MemModel}.
%
} \QuickQuizEnd

\QuickQuiz{}
	Why is it necessary to flush the cacheline from CPU~7's cache?
\QuickQuizAnswer{
	If the cacheline was not flushed from CPU~7's cache, then
	CPUs~0 and 7 might have different values for the same set
	of variables in the cacheline.
	This sort of incoherence would greatly complicate parallel
	software, and so hardware architects have been convinced to
	avoid it.
} \QuickQuizEnd

This simplified sequence is just the beginning of a discipline called
\emph{cache-coherency protocols}~\cite{Hennessy95a,DavidECuller1999,MiloMKMartin2012scale,DanielJSorin2011MemModel},
which is discussed in more detail in
Appendix~\ref{chp:app:whymb:Why Memory Barriers?}.
As can be seen in the sequence of events triggered by a CAS operation,
a single instruction can cause considerable protocol traffic, which
can significantly degrade your parallel program's performance.

Fortunately, if a given variable is being frequently read during a time
interval during which it is never updated, that variable can be replicated
across all CPUs' caches.
This replication permits all CPUs to enjoy extremely fast access to
this \emph{read-mostly} variable.
Chapter~\ref{chp:Deferred Processing} presents synchronization
mechanisms that take full advantage of this important hardware read-mostly
optimization.

\subsection{Costs of Operations}
\label{sec:cpu:Costs of Operations}

\begin{table}
\rowcolors{1}{}{lightgray}
\renewcommand*{\arraystretch}{1.1}
\centering\small
\begin{tabular}
  {
    l
    S[table-format = 9.1]
    S[table-format = 9.1]
  }
	\toprule
	Operation		& \multicolumn{1}{r}{Cost (ns)}
			& {\parbox[b]{.7in}{\raggedleft Ratio\\(cost/clock)}} \\
	\midrule
	Clock period		&           0.6	&           1.0 \\
	Best-case CAS		&          37.9	&          63.2 \\
	Best-case lock		&          65.6	&         109.3 \\
	Single cache miss	&         139.5	&         232.5 \\
	CAS cache miss		&         306.0	&         510.0 \\
	Comms Fabric		&       5 000	&       8 330	\\
	Global Comms		& 195 000 000	& 325 000 000   \\
	\bottomrule
\end{tabular}
\caption{Performance of Synchronization Mechanisms on 4-CPU 1.8\,GHz AMD Opteron 844 System}
\label{tab:cpu:Performance of Synchronization Mechanisms on 4-CPU 1.8GHz AMD Opteron 844 System}
\end{table}

The overheads of some common operations important to parallel programs are
displayed in
Table~\ref{tab:cpu:Performance of Synchronization Mechanisms on 4-CPU 1.8GHz AMD Opteron 844 System}.
This system's clock period rounds to 0.6\,ns.
Although it is not unusual for modern microprocessors to be able to
retire multiple instructions per clock period, the operations' costs are
nevertheless normalized to a clock period in the third column, labeled
``Ratio''.
The first thing to note about this table is the large values of many of
the ratios.

The best-case compare-and-swap (CAS) operation consumes almost forty
nanoseconds, a duration more than sixty times that of the clock period.
Here, ``best case'' means that the same CPU now performing the CAS operation
on a given variable was the last CPU to operate on this variable, so
that the corresponding cache line is already held in that CPU's cache.
Similarly, the best-case lock operation (a ``round trip'' pair consisting
of a lock acquisition followed by a lock release) consumes more than
sixty nanoseconds, or more than one hundred clock cycles.
Again, ``best case'' means that the data structure representing the
lock is already in the cache belonging to the CPU acquiring and releasing
the lock.
The lock operation is more expensive than CAS because it requires two
atomic operations on the lock data structure.

An operation that misses the cache consumes almost one hundred and forty
nanoseconds, or more than two hundred clock cycles.
The code used for this cache-miss measurement passes the cache line
back and forth between a pair of CPUs, so this cache miss is satisfied
not from memory, but rather from the other CPU's cache.
A CAS operation, which must look at the old value of the variable as
well as store a new value, consumes over three hundred nanoseconds, or
more than five hundred clock cycles.
Think about this a bit.
In the time required to do \emph{one} CAS operation, the CPU could have
executed more than \emph{five hundred} normal instructions.
This should demonstrate the limitations not only of fine-grained locking,
but of any other synchronization mechanism relying on fine-grained
global agreement.

\QuickQuiz{}
	Surely the hardware designers could be persuaded to improve
	this situation!
	Why have they been content with such abysmal performance
	for these single-instruction operations?
\QuickQuizAnswer{
	The hardware designers \emph{have} been working on this
	problem, and have consulted with no less a luminary than
	the physicist Stephen Hawking.
	Hawking's observation was that the hardware designers have
	two basic problems~\cite{BryanGardiner2007}:

	\begin{enumerate}
	\item	the finite speed of light, and
	\item	the atomic nature of matter.
	\end{enumerate}

\begin{table}
\rowcolors{1}{}{lightgray}
\renewcommand*{\arraystretch}{1.1}
\centering\small
\begin{tabular}
  {
    l
    S[table-format = 9.1]
    S[table-format = 9.1]
  }
	\toprule
	Operation		& \multicolumn{1}{r}{Cost (ns)}
			& {\parbox[b]{.7in}{\raggedleft Ratio\\(cost/clock)}} \\
	\midrule
	Clock period		&           0.4	&           1.0 \\
	``Best-case'' CAS	&          12.2	&          33.8 \\
	Best-case lock		&          25.6	&          71.2 \\
	Single cache miss	&          12.9	&          35.8 \\
	CAS cache miss		&           7.0	&          19.4 \\
	\midrule
	Off-Core		&		&		\\
	Single cache miss	&          31.2	&          86.6 \\
	CAS cache miss		&          31.2	&          86.5 \\
	\midrule
	Off-Socket		&		&		\\
	Single cache miss	&          92.4	&         256.7 \\
	CAS cache miss		&          95.9	&         266.4 \\
	Comms Fabric		&       2 600   &       7 220   \\
	Global Comms		& 195 000 000	& 542 000 000   \\
	\bottomrule
\end{tabular}
\caption{Performance of Synchronization Mechanisms on 16-CPU 2.8\,GHz Intel X5550 (Nehalem) System}
\label{tab:cpu:Performance of Synchronization Mechanisms on 16-CPU 2.8GHz Intel X5550 (Nehalem) System}
\end{table}

	The first problem limits raw speed, and the second limits
	miniaturization, which in turn limits frequency.
	And even this sidesteps the power-consumption issue that
	is currently holding production frequencies to well below
	10\,GHz.

	Nevertheless, some progress is being made, as may be seen
	by comparing
	Table~\ref{tab:cpu:Performance of Synchronization Mechanisms on 16-CPU 2.8GHz Intel X5550 (Nehalem) System}
	with
	Table~\ref{tab:cpu:Performance of Synchronization Mechanisms on 4-CPU 1.8GHz AMD Opteron 844 System}
	on
	page~\pageref{tab:cpu:Performance of Synchronization Mechanisms on 4-CPU 1.8GHz AMD Opteron 844 System}.
	Integration of hardware threads in a single core and multiple
	cores on a die have improved latencies greatly, at least within the
	confines of a single core or single die.
	There has been some improvement in overall system latency,
	but only by about a factor of two.
	Unfortunately, neither the speed of light nor the atomic nature
	of matter has changed much in the past few
	years~\cite{NoBugsHare2016CPUoperations}.

	Section~\ref{sec:cpu:Hardware Free Lunch?}
	looks at what else hardware designers might be
	able to do to ease the plight of parallel programmers.
} \QuickQuizEnd

I/O operations are even more expensive.
As shown in the ``Comms Fabric'' row,
high performance (and expensive!) communications fabric, such as
InfiniBand or any number of proprietary interconnects, has a latency
of roughly five microseconds for an end-to-end round trip, during which
time more than eight \emph{thousand} instructions might have been executed.
Standards-based communications networks often require some sort of
protocol processing, which further increases the latency.
Of course, geographic distance also increases latency, with the
speed-of-light through optical fiber latency around the world coming to
roughly 195 \emph{milliseconds}, or more than 300 million clock
cycles, as shown in the ``Global Comms'' row.

% Reference of Infiniband latency:
% http://www.hpcadvisorycouncil.com/events/2014/swiss-workshop/presos/Day_1/1_Mellanox.pdf
%     page 6/76 'Leading Interconnect, Leading Performance'

\QuickQuiz{}
	These numbers are insanely large!
	How can I possibly get my head around them?
\QuickQuizAnswer{
	Get a roll of toilet paper.
	In the USA, each roll will normally have somewhere around
	350-500 sheets.
	Tear off one sheet to represent a single clock cycle, setting it aside.
	Now unroll the rest of the roll.

	The resulting pile of toilet paper will likely represent a single
	CAS cache miss.

	For the more-expensive inter-system communications latencies,
	use several rolls (or multiple cases) of toilet paper to represent
	the communications latency.

	Important safety tip: make sure to account for the needs of
	those you live with when appropriating toilet paper!
} \QuickQuizEnd

\subsection{Hardware Optimizations}
\label{sec:cpu:Hardware Optimizations}

It is only natural to ask how the hardware is helping, and the answer
is ``Quite a bit!''

One hardware optimization is large cachelines.
This can provide a big performance boost, especially when software is
accessing memory sequentially.
For example, given a 64-byte cacheline and software accessing 64-bit
variables, the first access will still be slow due to speed-of-light
delays (if nothing else), but the remaining seven can be quite fast.
However, this optimization has a dark side, namely false sharing,
which happens when different variables in the same cacheline are
being updated by different CPUs, resulting in a high cache-miss rate.
Software can use the alignment directives available in many compilers
to avoid false sharing, and adding such directives is a common step
in tuning parallel software.

A second related hardware optimization is cache prefetching, in which
the hardware reacts to consecutive accesses by prefetching subsequent
cachelines, thereby evading speed-of-light delays for these
subsequent cachelines.
Of course, the hardware must use simple heuristics to determine when
to prefetch, and these heuristics can be fooled by the complex data-access
patterns in many applications.
Fortunately, some CPU families allow for this by providing special
prefetch instructions.
Unfortunately, the effectiveness of these instructions in the general
case is subject to some dispute.

A third hardware optimization is the store buffer, which allows a string
of store instructions to execute quickly even when the stores are to
non-consecutive addresses and when none of the needed cachelines are
present in the CPU's cache.
The dark side of this optimization is memory misordering, for which see
Chapter~\ref{chp:Advanced Synchronization: Memory Ordering}.

A fourth hardware optimization is speculative execution, which can
allow the hardware to make good use of the store buffers without
resulting in memory misordering.
The dark side of this optimization can be energy inefficiency and
lowered performance if the speculative execution goes awry and must
be rolled back and retried.
Worse yet, the advent of
Spectre and Meltdown~\cite{JannHorn2018MeltdownSpectre}
made it apparent that hardware speculation can also enable side-channel
attacks that defeat memory-protection hardware so as to allow unprivileged
processes to read memory that they should not have access to.
It is clear that the combination of speculative execution and cloud
computing needs more than a bit of rework!

A fifth hardware optimization is large caches, allowing individual
CPUs to operate on larger datasets without incurring expensive cache
misses.
Although large caches can degrade energy efficiency and cache-miss
latency, the ever-growing cache sizes on production microprocessors
attests to the power of this optimization.

A final hardware optimization is read-mostly replication, in which
data that is frequently read but rarely updated is present in all
CPUs' caches.
This optimization allows the read-mostly data to be accessed
exceedingly efficiently, and is the subject of
Chapter~\ref{chp:Deferred Processing}.

\begin{figure}[tb]
\centering
\resizebox{3in}{!}{\includegraphics{cartoons/Data-chasing-light-wave}}
\caption{Hardware and Software: On Same Side}
\ContributedBy{Figure}{fig:cpu:Hardware and Software: On Same Side}{Melissa Broussard}
\end{figure}

In short, hardware and software engineers are really fighting on the same
side, trying to make computers go fast despite the best efforts of the
laws of physics, as fancifully depicted in
Figure~\ref{fig:cpu:Hardware and Software: On Same Side}
where our data stream is trying its best to exceed the speed of light.
The next section discusses some additional things that the hardware engineers
might (or might not) be able to do, depending on how well recent
research translates to practice.
Software's contribution to this fight is outlined in the remaining chapters
of this book.
