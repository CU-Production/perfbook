% appendix/questions/parallelfaster.tex
% mainfile: ../../perfbook.tex
% SPDX-License-Identifier: CC-BY-SA-3.0

\section{Why Aren't Parallel Programs Always Faster?}
\label{sec:app:questions:Why Aren't Parallel Programs Always Faster?}
%
\epigraph{There is nothing quite so complicated as simplicity.}
	 {Charles Poore}

The short answer is ``because parallel execution often requires
communication, and communication is not free''.

For more information on this question, see
\cref{chp:Hardware and its Habits},
\cref{sec:count:Why Isn't Concurrent Counting Trivial?},
and especially
\cref{chp:Partitioning and Synchronization Design},
each of which present ways of slowing down your code by ineptly
parallelizing it.
Of course, much of this book deals with ways of ensuring that your
parallel programs really are faster than their sequential counterparts.

However, never forget that parallel programs can be quite fast while at
the same time being quite simple, with the example in
\cref{sec:toolsoftrade:Scripting Languages}
being a case in point.
Also never forget that parallel execution is but one optimization of many,
and there are programs for which other optimizations produce better results.
